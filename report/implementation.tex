\subsection{Setup}
We created a class \texttt{BaseLock} that encapsulates the behaviour of all the
locks with a simple interface, in order to write a framework that could test
all locks re-using the same structure and giving a constant overhead.
\begin{minted}[
	frame=lines,
	framesep=2mm,
	baselinestretch=.9,
	fontsize=\footnotesize,
	linenos,
	breaklines
	]
	{C}
class BaseLock {
  public:
    virtual void lock() = 0;
    virtual void unlock() = 0;
    virtual const char *get_name() = 0;
};
\end{minted}

All the implemented locks, extensions of the \texttt{BaseLock} class, are procedurally
tested from the \texttt{runLock} function.
Here, we instantiate the parallelization: we define a parallel part using the
\texttt{\#pragma omp parallel} instruction, then sync all threads with a barrier.
We then start measuring the time in each thread, since it might not be synchronised.
The last thread of the measured loops (i.e. when \texttt{counter == lock\_iterations})
will write the total time required for the loop to a shared variable, to then
compute the lock performance.

Furthermore in the critical section we count how many locks has each thread
using the \texttt{thread\_counter} variable, to then compute the fairness.

In the CS we also execute a constant load defined by the variable \texttt{cs\_iterations}
that will run an empty \texttt{while} loop for the required amount of cycles.

All these shared variables are then logged to a CSV outside of the lock execution
since file write might be a non constant bottleneck.

\begin{minted}[
	frame=lines,
	framesep=2mm,
	baselinestretch=.9,
	fontsize=\footnotesize,
	linenos,
	breaklines
	]
	{C}
int run_lock (
	std::ofstream& dataCollector,
	BaseLock *lock,
	int iteration,
	int nthreads,
	int lock_iterations,
	int cs_iterations
) {
	int counter = 0, duration;
	int thread_counter[THREAD_MAX] = {0};
	omp_set_num_threads(nthreads);

	#pragma omp parallel shared(counter, thread_counter, duration, lock_iterations)
	{
		int tid = omp_get_thread_num();
		std::chrono::time_point<std::chrono::high_resolution_clock> start, end;

		#pragma omp barrier
		start = std::chrono::high_resolution_clock::now();

		while(counter <= lock_iterations) {
			lock->lock();
			counter++;

			if (counter == lock_iterations) {
				end = std::chrono::high_resolution_clock::now();
				duration = std::chrono::duration_cast<std::chrono::microseconds>(end - start).count();
			}

			if (counter >= lock_iterations) {
				lock->unlock();
				break;
			}

			for(int cs = 0; cs < cs_iterations; cs++) {/*wait*/}

			thread_counter[tid]++;

			lock->unlock();
		}
	}

	dataCollector
		<< "\"" << lock->get_name() << "\","
		<< iteration << ","
		<< nthreads << ","
		<< cs_iterations << ","
		<< duration;

	for (int i = 0; i < nthreads; i++) {
		dataCollector << "," << thread_counter[i];
	}

	dataCollector << std::endl << std::flush;

	return EXIT_SUCCESS;
}
\end{minted}

The main function handles the file output and repeatedly runs this procedure,
in order to log the performance results for all the locks with a varying amount
of threads.

The data collected in the CSV is finally analysed using R where we extract the
statistics and produce the plots in this report.

\subsection{Filter Lock}
The Filter lock is a generalized version of the Peterson lock, it creates $n-1$ ``waiting rooms'', or levels,
that a thread must traverse before acquiring the lock. Levels satisfy two important properties:
\begin{itemize}
	\item At least one thread trying to enter level succeeds.
	\item If more than one thread is trying to enter level, then at least one is blocked
\end{itemize}

The Filter lock uses an n-element integer array (called \texttt{level[]}) to indicate the highest level that a thread is trying to enter. Each thread must pass $n-1$ levels to enter in the critical section and each level has its own victim to filter out one thread and excluding it from the next level.

The Filter lock satisfies mutual exclusion, deadlock freedom and starvation freedom properties but is not guaranteed to be fair since the threads can overtake others an arbitrary number of times (see exercise 12 from the book).

\begin{minted}[
	frame=lines,
	framesep=2mm,
	baselinestretch=.9,
	fontsize=\footnotesize,
	linenos,
	breaklines
	]
	{C}
#include "filterLock.hpp"

FilterLock::FilterLock (int n) {
	level = new int[n];
	victim = new int[n];
	this -> n = n;
}

FilterLock::~FilterLock () {
	delete[] level;
	delete[] victim;
}

void FilterLock::lock() {
	int me = omp_get_thread_num();
	for (int i = 1; i < n; i++) {
		level[me] = i;
		victim[i] = me;
		for( int j = 0; j < n; j++) {
			while ((j != me) && (level[j] >= i && victim[i] == me)) {}
		}
	}
}

void FilterLock::unlock() {
	int me = omp_get_thread_num();
	level[me] = 0;
}
\end{minted}

\subsection{Tournament tree of 2-thread Peterson locks}
Since the Peterson Lock works for a 2-thread implementation, a way to generalize it is to arrange a number of 2-thread locks in a binary tree. Suppose $n$ is a power of two. Each thread is assigned a leaf lock which it shares with one other thread. Each lock treats one thread as thread 0 and the other as thread 1.

In the tree-lock’s acquire method, the thread acquires every two-thread Peterson lock from that thread’s leaf to the root. The tree-lock’s release method for the tree-lock unlocks each of the 2-thread Peterson locks that thread has acquired, from the root back to its leaf.

The Tournament tree of 2-thread Peterson locks satisfies mutual exclusion, deadlock freedom and starvation freedom properties but it is not guaranteed to be fair since a thread can be delayed and overtaken an arbitrary number of times (see exercise 12 from the book).

\begin{minted}[
	frame=lines,
	framesep=2mm,
	baselinestretch=.9,
	fontsize=\footnotesize,
	linenos,
	breaklines
	]
	{C}
#include "petersonLock.hpp"

PetersonLock::PetersonLock(int n){
	numOfThreads = n;
	root = new PetersonNode(NULL, numOfThreads);

	std::vector<PetersonNode*> initList;
	initList.push_back(root);

	leaves = createTree(initList);
}

PetersonLock::~PetersonLock(){
}

void PetersonLock::lock() {
	int i = omp_get_thread_num();
	PetersonNode* currentNode = getLeaf(i);

	while (currentNode != NULL) {
		currentNode->lock();
		currentNode = currentNode->parent;
	}
}

void PetersonLock::unlock() {
	int i = omp_get_thread_num();
	PetersonNode* currentNode = getLeaf(i);

	while (currentNode != NULL) {
		currentNode->unlock();
		currentNode = currentNode->parent;
	}
}

// get leaf for a specific thread id, each lock shares two threads.
PetersonNode* PetersonLock::getLeaf(int num) {
	return leaves.at(num / 2);
}

std::vector<PetersonNode*> PetersonLock::createTree(std::vector<PetersonNode*> nodes) {
	if ((int)nodes.size() == numOfThreads / 2)
		return nodes;

	std::vector<PetersonNode*> currentLeaves;

	for (PetersonNode* node : nodes) {
		node->leftChild = new PetersonNode(node, numOfThreads);
		node->rightChild = new PetersonNode(node, numOfThreads);

		currentLeaves.push_back(node->leftChild);
		currentLeaves.push_back(node->rightChild);
	}
	return createTree(currentLeaves);
}
\end{minted}

\subsection{Herlihy-Shavit Bakery Lock}
The Herlihy-Shavit Bakery Lock is a modified version of the original Lamport's Bakery Lock implementation. In this lock, \texttt{flag[i]} is a Boolean flag indicating whether i wants to
enter the critical section, and \texttt{label[i]} is an integer that indicates the thread’s order when entering the ``bakery'', for each thread $i$.

Each time a thread acquires a lock, it generates a new \texttt{label[]} in two steps. First, it reads all the other threads’ labels in any order. Second, it reads all the other
threads’ labels one after the other (this can be done in some arbitrary order) and generates a label greater by one than the maximal label it read. We call the code from the raising of the flag (Line 19) to the writing of the new \texttt{label[]} (Line 30)
the doorway.
The lock establishes that thread’s order with respect to the other threads trying to acquire the lock. If two threads execute their doorways concurrently, they may read the same maximal label and pick the same new label. To break this symmetry, the algorithm uses a lexicographical ordering on pairs of \texttt{label[]} and thread ids at line 36, each thread read the labels one after the other in some arbitrary order until it determines that no
thread with a raised flag has a lexicographically smaller label/id pair, then it enters the critical section.

The Herlihy-Shavit Bakery Lock satisfies mutual exclusion, deadlock freedom and starvation freedom properties and it is also fair (first-come-first-serve, see Lemma 2.6.2 from the Book)

\begin{minted}[
	frame=lines,
	framesep=2mm,
	baselinestretch=.9,
	fontsize=\footnotesize,
	linenos,
	breaklines
	]
	{C}
#include "bakeryLock.hpp"

BakeryLock::BakeryLock (int num) {
	flag = new bool[num];
	label = new long long[num];
	for (int i = 0; i < num; i ++) {
		flag[i] = false;
		label[i] = 0;
	}
	n = num;
}
BakeryLock::~BakeryLock () {
	delete[] flag;
	delete[] label;
}

void BakeryLock::lock () {
	int i = omp_get_thread_num();
	flag[i] = true;
	long long max = label[0];
	for (int j = 1; j < n; j ++) {
		if (label[j] > max) {
			max = label[j];
		}
	}
	if (max == LLONG_MAX) {
		std::cout << "ERROR: Label Value Overflow" << std::endl;
		exit (-1);
	}
	label[i] = max + 1;

	for (int j = 0; j < n; j++) {
		if (i == j) {
			continue;
		}
		while (flag[j] && ((label[j] < label[i]) ||( label[i] == label[j] && j < i))) {};
	}
}

void BakeryLock::unlock () {
	flag[omp_get_thread_num()] = false;
}

\end{minted}

\subsection{Lamport Bakery Lock}
This is the original version of the Lamport's Bakery Lock\footnote{\url{http://lamport.azurewebsites.net/pubs/bakery.pdf}}.

Each time a thread acquires a lock, it generates a new \texttt{number[]} by reading all the other threads’ labels one after the other (this can be done in some arbitrary order) and generates a number greater by one than the maximal number it read. We call the code from the raising of the choosing flag (Line 20) to the writing of the new \texttt{number[]} (Line 22) the doorway section.

It establishes that thread’s order with respect to the other threads trying to acquire the lock. If two threads execute their doorways concurrently, they may read the same maximal label and pick the same new label. To break this symmetry, the algorithm uses a lexicographical ordering on pairs of \texttt{number[]} as already explained in the previous lock, then it enters the critical section.

The Lamport Bakery Lock satisfies mutual exclusion, deadlock freedom and starvation freedom properties and it is also fair, since if thread A executes the doorway before thread B then B is locked out while \texttt{number[A]} is greater than 0.

\begin{minted}[
	frame=lines,
	framesep=2mm,
	baselinestretch=.9,
	fontsize=\footnotesize,
	linenos,
	breaklines
	]
	{C}
#include "lamportLock.hpp"

LamportLock::LamportLock (int num) {
	choosing = new bool[num];
	number = new int[num];
	for (int i = 0; i < num; i ++) {
		choosing[i] = false;
		number[i] = 0;
	}
	n = num;
}

LamportLock::~LamportLock () {
	delete[] choosing;
	delete[] number;
}

void LamportLock::lock () {
	int i = omp_get_thread_num();
	choosing[i] = true;
	number[i] = findMax() + 1;
	choosing[i] = false;

	for (int j = 0; j < n; j++) {
		if (j == i)
			continue;

		while (choosing[j]) {}

		while (number[j] != 0 && (number[i] > number[j] || (number[i] == number[j] && i > j))) {}
	}
}

void LamportLock::unlock () {
	number[omp_get_thread_num()] = false;
}

int LamportLock::findMax() {
	int m = number[0];
	for (int i=1; i <n; ++i) {
		if (number[i] > m)
		m = number[i];
	}
	return m;
}

\end{minted}

\subsection{Boulangerie Lock}
The Boulangerie Lock \cite{MOSES201846} is a modified version of the Lamport Bakery Lock that applies two improvements:
\begin{itemize}
	\item Optimizing for low contention: if the thread $i$ has obtained \texttt{number[i] = 1} then the only processes $j$ that can ever have a better ticket are ones whose tid is smaller than $i$. It follows that when \texttt{number[i] = 1}, there is no need to perform the spinning section for values $j > i$. To avoid this form of unnecessary blocking, we add the control lines 36-41.

	\item Taking advantage of inconsistent reads: let's consider two threads $i$ and $j$, and that we perform read/write operation on safe registers. As long as $j$ is in the bakery it performs no writes on \texttt{number[j]} thus, \texttt{number[j]} is stable and all reads to it must return the same value. If $i$ reads two different values for \texttt{number[j]} while blocking during the spinning, it has proof that $j$ was on the outside at least during one of these reads. Since $i$ is the bakery section at that point, it follows that $i < j$ is true, and $i$ can stop blocking on $j$ and move on to test the next process.
\end{itemize}

The Boulangerie Lock satisfies mutual exclusion, deadlock freedom and starvation freedom properties and it is also fair, since if thread A executes the doorway before thread B then B is locked out while \texttt{number[A]} is greater than 0.


\begin{minted}[
	frame=lines,
	framesep=2mm,
	baselinestretch=.9,
	fontsize=\footnotesize,
	linenos,
	breaklines
	]
	{C}
#include "boulangerieLock.hpp"

BoulangerieLock::BoulangerieLock (int numThreads) {
	choosing = new bool[numThreads];
	number = new int[numThreads];
	num = new int[numThreads];
	for (int i = 0; i < numThreads; i ++) {
		choosing[i] = false;
		number[i] = 0;
		num[i] = 0;
	}
	n = numThreads;
}

BoulangerieLock::~BoulangerieLock () {
	delete[] choosing;
	delete[] number;
	delete[] num;
}

void BoulangerieLock::lock () {
	bool tmp_c = false;
	int *prev_n = nullptr;
	int *tmp_n = nullptr;
	int limit = n;
	int i = omp_get_thread_num();

	choosing[i] = true;
	for(int j=0; j<n; j++){
		num[i]=number[j];
	}
	num[i] = findMax() + 1;
	number[i] = num[i];
	choosing[i] = false;

	if(number[i]==1){
		limit = i;
	}
	else{
		limit = n;
	}

	for (int j = 0; j < limit; j++) {
		if (j == i)
			continue;

		do{
			tmp_c = choosing[j];
		} while(tmp_c);

		tmp_n = nullptr;

		do{
			prev_n = tmp_n;
			tmp_n = &number[j];
		} while (*tmp_n != 0 && (num[i] > *tmp_n || (num[i] == *tmp_n && i > j)) && (tmp_n == prev_n || prev_n == nullptr));
	}
}

void BoulangerieLock::unlock () {
	int i = omp_get_thread_num();
	num[i] = false;
	number[i] = false;
}

int BoulangerieLock::findMax() {
	int m = num[0];
	for (int k=1; k<n; ++k) {
		if (num[k] > m)
		m = num[k];
	}
	return m;
}

\end{minted}

\subsection{Base Locks}
For a baseline performance we consider three additional locks:
\begin{itemize}
	\item Test-and-Set Lock: has a single flag field per lock, the thread acquire lock by changing
	flag from false to true and it locks on success. To unlock it resets the flag. We know from the theory (see slides) that the performance of this lock is bad, due to the high memory contention. The lock is not fair and starvation free but it is fault tolerant.
	\item Test-and-Test-Set Lock: we test and set only if there is a chance of success. It has a better performance than TAS but memory contention and cache deletion problems are still present. The lock is not fair and starvation free but it is fault tolerant.
	\item Native OpenMP locks
\end{itemize}


\begin{minted}[
	frame=lines,
	framesep=2mm,
	baselinestretch=1.0,
	fontsize=\footnotesize,
	linenos,
	breaklines
	]
	{C}
#include "tas.hpp"

TestAndSetLock::TestAndSetLock(int n){
	state=false;
};

TestAndSetLock::~TestAndSetLock(){}

void TestAndSetLock::lock(){
	while(state.exchange(true)){}
};

void TestAndSetLock::unlock(){
	state.exchange(false);
};
\end{minted}

\begin{minted}[
	frame=lines,
	framesep=2mm,
	baselinestretch=1.0,
	fontsize=\footnotesize,
	linenos,
	breaklines
	]
	{C}
#include "ttas.hpp"

TestAndTestAndSetLock::TestAndTestAndSetLock(int n){
	state=false;
};

TestAndTestAndSetLock::~TestAndTestAndSetLock(){}

void TestAndTestAndSetLock::lock(){
	while(true){
		while(state){}
		if(!state.exchange(true))
		return;
	}
};

void TestAndTestAndSetLock::unlock(){
	state.exchange(false);
};
\end{minted}

\begin{minted}[
	frame=lines,
	framesep=2mm,
	baselinestretch=1.0,
	fontsize=\footnotesize,
	linenos,
	breaklines
	]
	{C}
#include <omp.h>

omp_lock_t mylock ;
omp_init_lock(&mylock ) ;

omp_set_lock(&writelock);

//critical section

omp_unset_lock(&writelock);
\end{minted}
